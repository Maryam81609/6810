\documentclass[11pt]{article}

\usepackage{color}

\newcommand{\q}[1]{\subsection*{Question {#1}}}
\renewcommand{\part}[1]{\paragraph*{{#1}.}}

\newcommand{\code}[1]{\textsf{#1}}
\newcommand{\answr}[1]{\textcolor{red}{#1}}
\newcommand{\inst}[2]{\textcolor{red}{{#1}{#2}}}

\title{Homework Assignment 1\\
	Performance Metrics\\
	CS/ECE 6810: Computer Architecture}
	
\author{Maryam Dabaghchian-u1078006}

%\date{\today}

\begin{document}
\maketitle

\q{1} 

\part{i} According to assumption all of the applications are equally likely to be 
executed by users, i.e., they are fairly executed. So, we compute sum of the 
weighted execution times. To that end, let us consider the computer system (BASE) 
as the reference machine. Thus, the weights of applications A, B, C, and D will 
respectively be: $\frac{1}{3}, \frac{2}{5}, 1, and \frac{1}{12}$. That results in 
the sum of weighted execution time for every machine as shown in 
Table~\ref{tbl:wext}. So, machine BASE has the least execution time. 

\begin{table}[h]
\center
\begin{tabular}{|l|r|}
	\hline
	 & \textbf{Weighted Sum} \\
	\hline
	\textbf{BASE} & $(3 \times \frac{1}{3}) + (\frac{5}{2} \times \frac{2}{5}) + 
	(1 \times 1) + (12 \times \frac{1}{12}) = 4.00$  \\
	\hline
	\textbf{NEW1} & $(7 \times \frac{1}{3}) + (3 \times \frac{2}{5}) + 
	(5 \times 1) + (1 \times \frac{1}{12}) = 8.62$ \\
	\hline
	\textbf{NEW2} & $(2 \times \frac{1}{3}) + (1 \times \frac{2}{5}) + 
	(3 \times 1) + (8 \times \frac{1}{12}) = 4.73$ \\
	\hline
	\textbf{NEW3} & $(1 \times \frac{1}{3}) + (3 \times \frac{2}{5}) + 
	(2 \times 1) + (13 \times \frac{1}{12}) = 4.62$ \\
	\hline
\end{tabular}
\caption{(i) Sum of weighted execution times.}
\label{tbl:wext}
\end{table}


\part{ii} We compute the weighted sum of energy for each machine similar to 
part (i). So, the weight of applications A, B, C, and D will respectively be: 
$\frac{1}{20}, \frac{1}{40}, \frac{1}{50}, and \frac{1}{15}$. As shown in 
Table~\ref{tbl:wenrgy}, machine NEW1 has the least energy consumption.

\begin{table}[h]
\center
\begin{tabular}{|l|r|}
	\hline
	 & \textbf{Total Energy} \\
	\hline
	\textbf{BASE} & $(20 \times \frac{1}{20}) + (40 \times \frac{1}{40}) + 
	(50 \times \frac{1}{50}) + (15 \times \frac{1}{15}) = 4$  \\
	\hline
	\textbf{NEW1} & $(10 \times \frac{1}{20}) + (30 \times \frac{1}{40}) + 
	(15 \times \frac{1}{50}) + (30 \times \frac{1}{15}) = 3.55$ \\
	\hline
	\textbf{NEW2} & $(30 \times \frac{1}{20}) + (60 \times \frac{1}{40}) + 
	(20 \times \frac{1}{50}) + (20 \times \frac{1}{15}) = 4.73$ \\
	\hline
	\textbf{NEW3} & $(70 \times \frac{1}{20}) + (35 \times \frac{1}{40}) + 
	(30 \times \frac{1}{50}) + (10 \times \frac{1}{15}) = 5.64$ \\
	\hline
\end{tabular}
\caption{(ii) Total energy consumption for every machine.}
\label{tbl:wenrgy}
\end{table}


\part{iii} Power shows the energy per time unit, i.e., $Power = 
\frac{Energy}{Time}$. Table~\ref{tbl:power} shows the power required by 
each application on all machines.

\begin{table}[h]
\center
\begin{tabular}{|l|c|c|c|c|}
	\hline
	 & \textbf{A} & \textbf{B} & \textbf{C} & \textbf{D} \\
	\hline
	\textbf{BASE} & $\frac{20}{3}$ & $\frac{40}{2.5}$ & $\frac{50}{1}$ 
	& $\frac{15}{12}$ \\
	\hline
	\textbf{NEW1} & $\frac{10}{7}$ & $\frac{30}{3}$ & $\frac{15}{5}$ 
	& $\frac{30}{1}$ \\
	\hline
	\textbf{NEW2} & $\frac{30}{2}$ & $\frac{60}{1}$ & $\frac{20}{3}$ 
	& $\frac{20}{8}$ \\
	\hline
	\textbf{NEW3} & $\frac{70}{1}$ & $\frac{35}{3}$ & $\frac{30}{2}$ 
	& $\frac{10}{13}$ \\
	\hline
\end{tabular}
\caption{(iii) Power required for every application.}
\label{tbl:power}
\end{table}

Based on the power for each application, we compute the weights similar to 
part (i) and (ii). So, the weight of applications A, B, C, and D will respectively 
be: $\frac{3}{20}$, $\frac{5}{80}$, $\frac{1}{50}$, and $\frac{12}{15}$. 
Therefore, according to weighted sum of power, as shown in 
Table~\ref{tbl:wpwr}, machine BASE consumes the least power.

\begin{table}[h]
\center
\begin{tabular}{|l|r|}
	\hline
	 & \textbf{Total Power} \\
	\hline
	\textbf{BASE} & $(\frac{20}{3} \times \frac{3}{20}) + (\frac{80}{5} \times 
	\frac{5}{80}) + (\frac{50}{1} \times \frac{1}{50}) + (\frac{15}{12} \times 
	\frac{12}{15})  = 4$  \\
	\hline
	\textbf{NEW1} & $(\frac{10}{7} \times \frac{3}{20}) + (\frac{30}{3} \times 
	\frac{5}{80}) + (\frac{15}{5} \times \frac{1}{50}) + 
	(\frac{30}{1} \times \frac{12}{15}) = 24.90$ \\
	\hline
	\textbf{NEW2} & $(\frac{30}{2} \times \frac{3}{20}) + (\frac{60}{1} \times 
	\frac{5}{80}) + (\frac{20}{3} \times \frac{1}{50}) + 
	(\frac{20}{8} \times \frac{12}{15}) = 8.13$ \\
	\hline
	\textbf{NEW3} & $(\frac{70}{1} \times \frac{3}{20}) + (\frac{35}{3} \times 
	\frac{5}{80}) + (\frac{30}{2} \times \frac{1}{50}) + 
	(\frac{10}{13} \times \frac{12}{15}) = 12.14$ \\
	\hline
\end{tabular}
\caption{(iii) Total power required for every machine.}
\label{tbl:wpwr}
\end{table}

\q{2}
\part{i} 
$
CPI = \sum \limits_{i=1}^{5} (Frequency_i \times Cycles_i) = 
(0.1 \times 2) + (0.05 \times 1) + (0.05 \times 2) + (0.3 \times 1) + 
(0.5 \times 4) = 2.65
$

$$
IPC = \frac{1}{CPI} = \frac{1}{2.65} = 0.38
$$

According to the assumption, 60\% of MULT instructions are merged with ADD 
instructions. That results in 0 number of ADD instructions and also 30\% 
($0.6 \times 0.5 = 0.3$) reduce in total number of instructions. So, we update 
the frequencies and cycle counts table accordingly as shown in 
Table~\ref{tbl:newfreq}.

\begin{table}[h]
\center
\begin{tabular}{|c|c|c|c|c|c|c|}
	\hline
	 & \textbf{Load} & \textbf{Store} & \textbf{Branch} & \textbf{ADD} & 
	 \textbf{MULT} & \textbf{FMAD} \\
	\hline
	\textbf{Frequency} & 14\% & 7\% & 7\% & 0\% & 29\% & 43\% \\
	\hline
	\textbf{Cycles} & 2 & 1 & 2 & 1 & 4 & 4 \\
	\hline
\end{tabular}
\caption{Frequencies and cycle counts for instructions for the new processor.}
\label{tbl:newfreq}
\end{table}

Now, we compute IPC based on new measures in Table~\ref{tbl:newfreq}.

$
CPI = \sum \limits_{i=1}^{6} (Frequency_i \times Cycles_i) = 
(0.14 \times 2) + (0.07 \times 1) + (0.07 \times 2) + (0 \times 1) + 
(0.29 \times 4) + (0.43 \times 4) = 3.37
$

$$
IPC = \frac{1}{CPI} = \frac{1}{3.37} = 0.30
$$

\part{ii} 
$CPUTime = CPI \times IC \times CT$

$CPUTime_{old} = 2.65 \times 1 \times CT$

$CPUTime_{new} = 3.37 \times 0.7 \times CT$

\[
Speedup = \frac{CPUTime_{old}}{CPUTime_{new}} = \frac{2.65 CT}{(3.37 
\times 0.7) CT} = 1.13
\]


\q{3} Table~\ref{tbl:amdahl} summarizes the question. 

\begin{table}[h]
\center
\begin{tabular}{|c|c|c|c|c|}
	\hline
	 & \textbf{WI} & \textbf{Graphics} & \textbf{CPU} & \textbf{Other} \\
	\hline
	\textbf{Fraction} & 50\% & 20\% & 10\% & 20\% \\
	\hline
\end{tabular}
\caption{Energy consumption by each unit.}
\label{tbl:amdahl}
\end{table}

Also, amdahl's law is as follows:

\[
Improvement_{overall} = \frac{1}{(1-Fraction_{enhanced}) + \frac{Fraction_{enhanced}}{Improvement_{enhanced}}}.
\]
\part{i} Reducing WI energy by 10\%, i.e., $energy'_{WI} = 0.9 energy_{WI}$. 
So, $Improvement_{enhanced} = \frac{1}{0.9}$, and thereby:

\[
Improvement_{overall} = \frac{1}{(1-0.5) + \frac{0.5}{1/0.9}} = 
\frac{1}{0.95} = 1.05
\]

\part{ii} Reducing CPU energy by 60\%, i.e., $energy'_{CPU} = 0.4 energy_{CPU}$. 
So, $Improvement_{enhanced} = \frac{1}{0.4}$, and thereby:

\[
Improvement_{overall} = \frac{1}{(1-0.1) + \frac{0.1}{1/0.4}} = 
\frac{1}{0.94} = 1.06
\]

\part{iii} Reducing Graphics energy by 50\%, i.e., $energy'_{G} = 0.5 energy_{G}$. 
So, $Improvement_{enhanced} = \frac{1}{0.5}$, and thereby:

\[
Improvement_{overall} = \frac{1}{(1-0.2) + \frac{0.2}{1/0.5}} = 
\frac{1}{0.9} = 1.11
\]

Therefore, reducing the graphics energy by 50\% is a better energy optimization.

\q{4} $Power_{dynamic} = 70^w,  Power_{static} = 30^w,  Time = 15^s$

\part{i} $Energy = (Power_{dynamic} + Power_{static}) \times Time$.

\[
Energy = (70 + 30) \times 15 = 1500 ^J
\]

\part{ii} Scaling down the frequency (by 30\%) affects the dynamic power and time. 

$Power_{dynamic} = Activity x Capacitance x Voltage^2 x Frequency$

\[
Power'_{dynamic} = 0.7 Power_{dynamic} = 0.7 \times 70 = 49^w
\]
\[
Time' = \frac{1}{0.7} \times Time = \frac{15}{0.7}^s
\]
\[
Energy = (49 + 30) \times \frac{15}{0.7} = 1692.86 ^J
\]

\part{iii} Reducing the voltage affects both static and dynamic powers. In addition,  
reducing the frequency affects the dynamic power and time. 

$Power_{dynamic} = Activity \times Capacitance \times Voltage^2 
\times Frequency$

$Power_{static} = Voltage \times CurrentStatic$

\[
Power'_{dynamic} = 0.7 \times 0.7^2 \times Power_{dynamic} = 
0.7^3 \times 70 = 24.01^w
\]
\[
Power'_{static} = 0.7 \times Power_{static} = 0.7 \times 30 = 21^w
\]
\[
Time' = \frac{1}{0.7} \times Time = \frac{15}{0.7}^s
\]
\[
Energy = (24.01 + 21) \times \frac{15}{0.7} = 964.5^J 
\]


\q{5}

LOAD R5,  6000(R0) \\
\inst{1}{:} \answr{Reg[5] = Mem[6000+Reg[0]] = Mem[7000] = 1} \\
ADD R4, (R4) \\
\inst{2}{:} \answr{Reg[4] = Reg[4] + Mem[Reg[4]] = 6000 + 12 = 6012} \\
SUB R2, R1 \\
\inst{3}{:} \answr{Reg[2] = Reg[2] - Reg[1] = 99 - 25 = 74} \\
LOAD R6, @(R0) \\
\inst{4}{:} \answr{Reg[6] = Mem[Mem[Reg[0]]] = 33} \\
ADD R6, R4 \\
\inst{5}{:} \answr{Reg[6] = Reg[6] + Reg[4] = 6045} \\
SUB R5, R6 \\
\inst{6}{:} \answr{Reg[5] = Reg[5] - Reg[6] = -6044} \\
ADD R2, R5 \\
\inst{7}{:} \answr{Reg[2] = Reg[2] + Reg[5] = -5970} \\
ADD R2, (R3+R0) \\
\inst{8}{:} \answr{Reg[2] = Reg[2] + Mem[Reg[3] + Reg[0]] = -5970 + 71 = -5899}


\begin{table}[h]
\center
\begin{tabular}{|c|c|c|c|c|c|c|c|}
	\hline
	Instruction\# & R0 & R1 & R2 & R3 & R4 & R5 & R6 \\
	\hline
	\answr{Initial state} & 1000 & 25 & 99 & 4000 & 6000 & & \\
	\hline 
	\inst{1}{} & 1000 & 25 & 99 & 4000 & 6000 & \answr{1} & \\
	\hline
	\inst{2}{} & 1000 & 25 & 99 & 4000 & \answr{6012} & \answr{1} & \\
	\hline
	\inst{3}{} & 1000 & 25 & \answr{74} & 4000 & \answr{6012} & \answr{1} & \\
	\hline
	\inst{4}{} & 1000 & 25 & \answr{74} & 4000 & \answr{6012} & \answr{1} & \answr{33} \\
	\hline
	\inst{5}{} & 1000 & 25 & \answr{74} & 4000 & \answr{6012} & \answr{1} & \answr{6045} \\
	\hline
	\inst{6}{} & 1000 & 25 & \answr{74} & 4000 & \answr{6012} & \answr{-6044} & \answr{6045} \\
	\hline
	\inst{7}{} & 1000 & 25 & \answr{-5970} & 4000 & \answr{6012} & \answr{-6044} & \answr{6045} \\
	\hline
	\inst{8}{} & 1000 & 25 & \answr{-5899} & 4000 & \answr{6012} & \answr{-6044} & \answr{6045} \\
	\hline
\end{tabular}
\caption{Register values.}
\end{table}
\end{document}