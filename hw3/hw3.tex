\documentclass[11pt]{article}

\usepackage{geometry}
%\geometry{hoffset=-0.6in}
\usepackage[table]{xcolor}
\usepackage{color}


\newcommand{\q}[1]{\subsection*{Question {#1}}}
\renewcommand{\part}[1]{\paragraph*{{#1}.}}

\newcommand{\code}[1]{\textsf{#1}}

%\newcommand{\answr}[1]{\textcolor{blue}{#1}}
%\newcommand{\inst}[2]{\textcolor{blue}{{#1}{#2}}}

\newcommand{\stall}{\textcolor{red}{\textbf{Stall}}}
\newcommand{\cellstall}{\cellcolor{red!25}\textcolor{black}{\textbf{Stall}}}

\title{Homework Assignment 3\\
	Instruction Level Parallelism\\
	CS/ECE 6810: Computer Architecture}
	
\author{Maryam Dabaghchian-u1078006}

\date{\today}

\begin{document}
\maketitle

\q{1} 

\part{i} The following shows the original assembly code with stalls at the left hand side and the optimized schedule without unrolling at the right hand side. 
To optimize the code, we move \code{DADDUI R1, R1, \#-8} to where the first stall is, and then the other two \code{DADDUI}s for \code{R2} and \code{R3} to the where the next two stalls are. Finally, we move the branch instruction 1 step before store, so that we eliminate the last \code{NOP} resulting from the branch instruction, i.e., the store instruction executes at the \code{NOP} for branch instruction cycle. In the optimized schedule, since the memory address in \code{R3} has already been decreased, we should used offset 8 when storing \code{F3} value at the previous memory location denoted by \code{R3}.

\begin{table}[h]
\center
\small
\begin{tabular}{lcccl}
	 \code{Loop:} \code{L.D F1, 0(R1)} & & & &  \\
	 \code{L.D F2, 0(R2)}  & & & & \\
	 \stall &  & & &  \code{Loop:} \code{L.D F1, 0(R1)} \\
	 \code{MUL F3, F1, F2}  & & & & \code{L.D F2, 0(R2)} \\
	 \stall &  & & & \code{DADDUI R1, R1, \#-8} \\
	 \stall & & & & \code{MUL F3, F1, F2}  \\
	 \stall & & & & \code{DADDUI R2, R2, \#-8}  \\
	 \stall & & $\Rightarrow$ & & \code{DADDUI R3, R3, \#-8}  \\
	 \code{S.D F3, 0(R3)} & & & & \stall  \\
	 \code{DADDUI R1, R1, \#-8} & & & &  \code{BNE R1, R4, Loop} \\
	 \code{DADDUI R2, R2, \#-8} & & & &  \code{S.D F3, 8(R3)} \\
	 \code{DADDUI R3, R3, \#-8} & & & &   \\
	 \code{BNE R1, R4, Loop} & & & &  \\
	 \code{NOP} & & & &  \\
	 Q1(i): Original schedule with stalls. & & & & Q1(i): Optimized schedule without unrolling.
\end{tabular}
\label{tbl:q1p1}
\end{table}


\part{ii} 
For the unrolled second iteration, we rename \code{F1} to \code{F4}, \code{F2} to \code{F5}, and \code{F3} to \code{F6}. Then, the original schedule obtained by only unrolling twice and its optimized version are shown on the left hand side and the right hand side, respectively, as follows. To optimize the code, we move \code{L.D F4, -8(R1)} up to fill the first stall cycle, and then move the other load \code{L.D F5, -8(R2)} up to fill the second stall cycle, which is right after \code{MUL F3, F1, F2} instruction, and so forth. While execution store of the first iteration, we should use offset 16, as the memory address in \code{R3} has already been decreased, so that we store writes to the correct location. Similarly, for the store of the second iteration we should use offset 8.

\begin{table}[h]
\center
\small
\begin{tabular}{lcccl}
	  \code{Loop:} \code{L.D F1, 0(R1)} & & & &  \\
	 \code{L.D F2, 0(R2)} & & & &  \\
	 \stall & & & &   \\
	 \code{MUL F3, F1, F2} & & & &  \\
	 \stall & & & &  \\
	 \stall & & & &  \code{Loop:} \code{L.D F1, 0(R1)}  \\
	 \stall & & & &  \code{L.D F2, 0(R2)} \\
	 \stall & & & & \code{L.D F4, -8(R1)}  \\
	 \code{S.D F3, 0(R3)} & & & & \code{MUL F3, F1, F2}  \\
	 \code{L.D F4, -8(R1)} & & & &  \code{L.D F5, -8(R2)}  \\
	 \code{L.D F5, -8(R2)} & & $\Rightarrow$ & & \code{DADDUI R1, R1, \#-16} \\
	 \stall & & & &  \code{MUL F6, F4, F5} \\
	 \code{MUL F6, F4, F5} & & & & \code{DADDUI R2, R2, \#-16}  \\
	 \stall & & & &  \code{DADDUI R3, R3, \#-16} \\
	 \stall  & & & & \code{S.D F3, 16(R3)} \\
	 \stall & & & & \code{BNE R1, R4, Loop}  \\
	 \stall & & & & \code{S.D F3, 8(R3)}  \\	 
	 \code{S.D F6, -8(R3)} & & & &  \\
	 \code{DADDUI R1, R1, \#-16} & & & &  \\
	 \code{DADDUI R2, R2, \#-16} & & & &  \\
	 \code{DADDUI R3, R3, \#-16} & & & &  \\
	 \code{BNE R1, R4, Loop} & & & &  \\
	 \code{NOP} & & & & \\
	 Q1(ii): Unrolled schedule. & & & & Q1(ii): Optimized unrolled schedule.
\end{tabular}
\label{tbl:q1p2}
\end{table}

\part{iii} According to the slide 21 at the 8$^{th}$ lecture, in iteration \textit{i} the \code{S.D} instruction from two previous iteration, i.e., \code{S.D[i+2]} (because \textit{i} is decreasing), and the \code{MUL} instruction from the previous iteration, i.e., \code{MUL[i+1]} and two load instructions from the current iteration, i.e., \code{L.D[i]} form the loop body as shown in the following. In the software pipelined version, \code{MUL F3, F1, F2} uses the values of \code{F1} and \code{F2} from the previous iteration, as they are loaded with the new values after \code{MUL} instruction executes. Likewise, in the instruction \code{S.D F3, 16(R3)}, the value of \code{F3} is from the previous iteration, which itself is computed based on the values of \code{F1} and \code{F2} in two previous iteration. Therefore, \code{S.D} instruction is attributed to two previous iteration, which thereby should write to the memory location at \code{R3} with offset 16.

\begin{table}[h]
\center
\small
\begin{tabular}{lcccl}
	  \code{Loop:} \code{L.D F1, 0(R1)} & & & &  \code{Loop:} \code{S.D F3, 16(R3)} \\
	 \code{L.D F2, 0(R2)}  & & & & \code{MUL F3, F1, F2}\\
	 \code{MUL F3, F1, F2}  & & & & \code{L.D F1, 0(R1)} \\
	 \code{S.D F3, 0(R3)} & & & & \code{L.D F2, 0(R2)} \\
	 \code{DADDUI R1, R1, \#-8} & & $\Rightarrow$ & & \code{DADDUI R1, R1, \#-8} \\
	 \code{DADDUI R2, R2, \#-8} & & & & \code{DADDUI R2, R2, \#-8} \\
	 \code{DADDUI R3, R3, \#-8} & & & & \code{DADDUI R3, R3, \#-8} \\
	 \code{BNE R1, R4, Loop} & & & & \code{BNE R1, R4, Loop} \\ 
	 \code{NOP} & & & & \code{NOP} \\
	 Q1(iii): Original loop body. & & & & Q1(iii): Software pipelined loop body.
\end{tabular}
\label{tbl:q1p3}
\end{table}

\q{2} We reorder instructions, preserving the dependency and waiting time between instructions, to eliminate the stalls. The following shows the original schedule with stall cycles at the left hand side and the optimized schedule with no stall cycle at the right hand side. 

\begin{table}[h]
\center
\small
\begin{tabular}{lcccl}
	  \code{Loop:} \code{L.D F2, 0(R2)} & & & & \\
	 \code{L.D F3, 0(R3)} & & & & \\
	 \stall & & & &\\
	 \code{MULT.D F1, F2, F3} & & & &  \code{Loop:} \code{L.D F2, 0(R2)} \\
	 \code{L.D F4, 0(R4)} & & & & \code{L.D F3, 0(R3)} \\
	 \stall & & & & \code{L.D F4, 0(R4)} \\
	 \code{ADD.D F5, F4, F1} & & & & \code{MULT.D F1, F2, F3} \\
	 \stall & & & & \code{ADD.D F5, F4, F1} \\
	 \stall & & $\Rightarrow$ & & \code{DADDUI R2, R2, \#-8} \\
	 \stall & & & & \code{DADDUI R3, R3, \#-8} \\
	 \stall & & & & \code{DADDUI R4, R4, \#-8} \\
	 \code{S.D F5, 0(R5)} & & & & \code{DADDUI R5, R5, \#-8} \\
	 \code{DADDUI R2, R2, \#-8} & & & & \code{BNE R2, R1, Loop} \\
	 \code{DADDUI R3, R3, \#-8} & & & & \code{S.D F5, 8(R5)} \\
	 \code{DADDUI R4, R4, \#-8} & & & &\\
	 \code{DADDUI R5, R5, \#-8} & & & & \\
	 \code{BNE R2, R1, Loop} & & & &\\
	 \code{NOP} & & & &\\
	 Q2: Original schedule with stalls. & & & & Q2: Optimized schedule without unrolling.
\end{tabular}
\label{tbl:q2}
\end{table}


\q{3} 
\part{i} 

\begin{table}[h]
\center
\small
\begin{tabular}{llcll}
	 Int Pipeline & FP Pipeline & & Int Pipeline & FP Pipeline \\
	 \code{Loop:} \code{L.D F1, 0(R1)} &  & & & \\
	 \cellstall &  & & & \\
	 \stall & \code{F.MUL F3, F1, F2} & & \code{Loop:} \code{L.D F1, 0(R1)} &  \\
	 \cellstall & \cellstall & & \code{DADDUI R1, R1, \#-8} &  \\
	 \stall & \code{F.ADD F5, F3, F4} & & \code{DADDUI R2, R2, \#-8} & \code{F.MUL F3, F1, F2} \\
	 \cellstall &  & & \cellstall & \cellstall \\
	 \cellstall &  & $\Rightarrow$ & \stall & \code{F.ADD F5, F3, F4} \\
	 \cellstall &  & & \cellstall &  \\
	 \cellstall &  & &  \cellstall &   \\
	 \code{S.D F5, 0(R2)} &  & & \cellstall &  \\
	 \code{DADDUI R1, R1, \#-8} &  & &  \code{BNE R1, R3, Loop} &  \\
	 \code{DADDUI R2, R2, \#-8} &  & &  \code{S.D F5, 8(R2)} &  \\
	 \code{BNE R1, R3, Loop}  &  & &  &  \\
	 \code{NOP} &  & & & \\
	 \multicolumn{2}{l}{Q3(i): Original schedule.} & & \multicolumn{2}{l}{Q3(i): Optimized schedule without unrolling.} 
\end{tabular}
\label{tbl:q3-p1}
\end{table}


\part{ii} 

\begin{table}[h]
\center
\small
\begin{tabular}{llcll}
	 Int Pipeline & FP Pipeline & & Int Pipeline & FP Pipeline \\
	 \code{Loop:} \code{L.D F11, 0(R1)} &  & & & \\
	 \stall &  & & & \\
	 \stall & \code{F.MUL F13, F11, F12} & & & \\
	 \cellstall & \cellstall & & & \\
	 \stall & \code{F.ADD F15, F13, F14} & & \code{Loop:} \code{L.D F11, 0(R1)} &  \\
	 \cellstall & \cellstall & & \code{L.D F21, -8(R1)} &  \\
	 \cellstall & \cellstall &  & \code{L.D F31, -16(R1)} & \code{F.MUL F13, F11, F12} \\
	 \cellstall & \cellstall & & \code{L.D F41, -24(R1)} & \code{F.MUL F23, F21, F22} \\
	 \cellstall & \cellstall & & \code{L.D F51, -32(R1)} & \code{F.MUL F33, F31, F32} \\
	 \code{S.D F15, 0(R2)} & \stall & & \code{DADDUI R1, R1, \#-40} & \code{F.MUL F43, F41, F42} \\
	 \code{L.D F21, -8(R1)} & \stall & & \code{DADDUI R2, R2, \#-40} & \code{F.MUL F53, F51, F52} \\
	 \cellstall & \cellstall & & \stall & \code{F.ADD F15, F13, F14} \\ 
	 \stall & \code{F.MUL F23, F21, F22} & & \stall & \code{F.ADD F25, F23, F24} \\
	 \cellstall & \cellstall & & \stall & \code{F.ADD F35, F33, F34} \\
	 \stall & \code{F.ADD F25, F23, F24} & & \stall & \code{F.ADD F45, F43, F44} \\
	 \cellstall & \cellstall & & \stall & \code{F.ADD F55, F53, F54} \\
	 \cellstall & \cellstall & $\Rightarrow$ & \code{S.D F15, 40(R2)} &   \\
	 \cellstall & \cellstall & & \code{S.D F25, 32(R2)} &  \\ 
	 \cellstall & \cellstall & & \code{S.D F35, 24(R2)} &  \\
	 \code{S.D F25, -8(R2)} & \stall & & \code{S.D F45, 16(R2)} &  \\
	 \code{L.D F31, -16(R1)} & \stall & & \code{BNE R1, R3, Loop} &  \\ 
	 \cellstall & \cellstall & &  \code{S.D F55, 8(R2)} &  \\	
	 \stall & \code{F.MUL F33, F31, F32} & &  &  \\ 
	 \cellstall & \cellstall & & & \\
	 \stall & \code{F.ADD F35, F33, F34} & & & \\
	 \cellstall &  & & & \\
	 \cellstall &  &  &  & \\
	 \cellstall &  & &  &  \\
	 \cellstall &  & &   &   \\
	 \code{S.D F35, -16(R2)} &  & &  &  \\ 
	 \code{L.D F41, -24(R1)} & \stall & & & \\	
	 \cellstall & \cellstall & & & \\
	 \stall & \code{F.MUL F43, F41, F42} & & & \\
	 \cellstall & \cellstall & & &  \\
	 \stall & \code{F.ADD F45, F43, F44} & & & \\
	 \cellstall &  & & & \\
	 \cellstall &  &  &  & \\
	 \cellstall &  & &  &  \\
	 \cellstall &  & &   &   \\
	 \code{S.D F45, -24(R2)} &  & &  &  \\  
	 \multicolumn{2}{l}{Q3(ii): Unrolled schedule for 5 times.} & & \multicolumn{2}{l}{Q3(ii): Optimized schedule with unrolling 5 times.} 
\end{tabular}
\label{tbl:q3-p1-1}
\end{table}

\begin{table}[h]
\center
\small
\begin{tabular}{llcll}
	 Int Pipeline & FP Pipeline & & Int Pipeline & FP Pipeline \\ 
	 \code{L.D F51, -32(R1)} & \stall & & & \\	
	 \cellstall & \cellstall & & & \\
	 \stall & \code{F.MUL F53, F51, F52} & & & \\
	 \cellstall & \cellstall & & &  \\
	 \stall & \code{F.ADD F55, F53, F54} & & & \\
	 \cellstall &  & & & \\
	 \cellstall &  &  &  & \\
	 \cellstall &  & &  &  \\
	 \cellstall &  & &   &   \\
	 \code{S.D F55, -32(R2)} &  & &  &  \\ 
	 \code{DADDUI R1, R1, \#-40} &  & &  & \\
	 \code{DADDUI R2, R2, \#-40} &  & &  & \\
	 \code{BNE R1, R3, Loop}  &  & &  &  \\
	 \code{NOP} &  & & & \\
	 \multicolumn{2}{l}{Q3(ii): Unrolled schedule for 5 times.} & & \multicolumn{2}{l}{Q3(ii): Optimized schedule with unrolling 5 times.} 
\end{tabular}
\label{tbl:q3-p1-2}
\end{table}

\end{document}